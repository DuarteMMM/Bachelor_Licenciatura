\section{Conclusion} \label{sec:conclusion}

In this laboratory assignment, a bandpass filter using an operational amplifier (OP-AMP) was successfully implemented and studied in three different ways: by using a theoretical model and the Octave math tool, by implementing the circuit in the laboratory and by using Ngspice simulations. A central frequency very close to the desired 1 kHz has been obtained, as well as a gain in the passband quite close to the desired 40 dB. Taking into account the restrictions imposed on the number of components and the values of the respective resistances and capacitances, an acceptable merit was obtained. Even though the merit figure was improved as much as possible, these restrictions and the high monetary cost of the OP-AMP made it quite difficult to further increase it.
\par
The theoretical and simulation's results were very similar to each other, except the gain and phase plots. These differences are due to the fact that an ideal OP-AMP model was considered in the Theoretical Analysis, but a quite complex model was used in the Ngspice simulations. This complex model used two capacitors, thus the simulation's results included two extra poles in the transfer function. Apart from this, the ideal OP-AMP model was able to provide good theoretical results.
