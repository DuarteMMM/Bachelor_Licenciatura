%
% Compilar com
%    pdflatex Aula_2019_26.tex 
%

\documentclass[12pt]{beamer}

\usepackage[latin1,utf8]{inputenc}
\usepackage[portuges]{babel}
\usepackage{amsmath}
\usepackage{bm}

%\usepackage[T1]{fontenc}

\usetheme{Luebeck}
%\usetheme{Luebeck}
%\usetheme{Boadilla}
%\usetheme{Madrid}
%\usetheme{default}
%\usetheme{Hannover}
% 2008-2013 \usetheme{Luebeck}

\useinnertheme{default}
%\useinnertheme{rectangles}
%\useinnertheme{inmargin}

\useoutertheme{default}
%\useoutertheme{shadow}
%\useoutertheme{smoothtree}

%\usefonttheme{default}
%\usefonttheme[hoptionsi]{serif}
%\usefonttheme{structuresmallcapsserif}
%\usefonttheme{structureitalicserif}

%\usecolortheme{default}
%\usecolortheme{albatross}
\usecolortheme{lily}
%2008-2012 \usecolortheme{lily}

% Texto normal

%\setbeamercolor{normal text}{bg=red!20} % Cor do fundo 
\setbeamercolor{normal text}{fg=black}  % Cor do texto

%  frametitle

\setbeamercolor{frametitle}{fg=blue} % MidnightBlue \color[RGB]{1,0,0}
%\setbeamercolor{frametitle}{fg=rgb={.2,.2,.7}} % MidnightBlue
%\setbeamerfont{frametitle}{series=\bfseries}
\setbeamerfont{frametitle}{size=\large}
\setbeamertemplate{frametitle}
{
\begin{centering}
%\color{blue}
\textbf{\insertframetitle}
\par
\end{centering}
}


% Definição de barras à direita ou à esquerda:

%\setbeamersize{sidebar width left=1cm}
%\setbeamersize{sidebar width right=1cm}

\setbeamersize{text margin left=0.5cm}
\setbeamersize{text margin right=0.2cm}


%
% Definição de cores
%
\definecolor{DarkRed}{rgb}{0.54,0,0}
\definecolor{brown4}{rgb}{0.54,0.137,0.137}
\definecolor{green4}{rgb}{0,0.543,0}
\definecolor{DarkGreen}{rgb}{0,0.543,0}
%\definecolor{colorteste}{cmyk}{0.8,0.85,1}
%\definecolor{redblack}{rgb}{0.8,0.85,1}

%
% Notas:
%
%  \textbf{}, \textit, \textsl, \textrm, \textsf, \color,
%  \alert, \structure.

%


\title[Programação - 26ª Aula (11.12.2019)]{\bf \Large{Programação} \\ 
\large{Mestrado em Engenharia Física Tecnológica}}
\author[Mestrado em Engenharia Física Tecnológica]{Samuel M. Eleutério \\ 
\texttt{sme@tecnico.ulisboa.pt}}
\date[IST-UTL]{Departamento de Física \\ Instituto Superior Técnico \\ 
Universidade de Lisboa}

\begin{document}

\begin{frame}
\center{\bf \large{26ª Aula - Tratamento de Texto Científico
\vskip 2mm\par \TeX, \LaTeX}}
\titlepage
\end{frame}

\begin{frame}
\frametitle{{\color{blue}\bf Introdução}}
\begin{itemize}
\item<1->{\color{blue}\bf \TeX}\ ({\color{blue}\bf TeX}) é um sistema de 
escrita de {\color{red}\bf texto científico}. 
Foi desenvolvido por {\color{blue}\bf Donald Knuth}, então professor
na Stanford University (Ciência da Computação).
\item<2->Foram as provas tipográficas da segunda
edição do seu livro ''{\color{blue}\bf The Art of Computer Programming}'' 
em 1977 que determinaram a sua decisão de desenvolver o que veio a
ser o {\color{blue}\bf \TeX}.
\item<3->Desde então, um número cada vez maior de 
{\color{blue}\bf textos científicos} passaram a ser 
{\color{DarkRed}\bf dactilografados} directamente 
pelos {\color{red}\bf autores}.
\item<4->Actualmente o {\color{blue}\bf \TeX} é a {\color{red}\bf base de escrita} 
incontornável de qualquer
{\color{DarkGreen}\bf texto científico} nas áreas das {\color{DarkRed}\bf Ciências} 
e das {\color{DarkRed}\bf Engenharias}.
\item<5->O nome {\color{blue}\bf TeX} vem do grego 
{\color{red}\boldmath $\tau\acute{\epsilon}\chi\nu\eta$}
('{\color{DarkRed}\bf arte}', de onde vem a palavra '{\color{DarkRed}\bf técnica}').
Pronuncia-se como {\color{DarkGreen}\bf /'t${\bm \epsilon}$x/} ou como 
{\color{DarkGreen}\bf /'t{${\bm \epsilon}$}k/} 
em que o '{\color{red}\bf X}' se pronuncia como 
'lo{\color{red}\bf ch}' ou como ('{\color{red}\bf q}ueen').
\end{itemize}
\end{frame}

\begin{frame}
\frametitle{{\color{blue}\bf Introdução - \TeX}}
\begin{itemize}
\item<1->O {\color{blue}\bf TeX} é uma linguagem baseada em 
'{\color{DarkGreen}\bf macros}' e '{\color{DarkGreen}\bf tokens}', 
sendo muitos desses comandos {\color{DarkRed}\bf desmanchados em tempo real}.
\item<2->{\color{blue}\bf Knuth} muniu inicialmente o {\color{blue}\bf TeX} 
de cerca de {\color{red}\bf 600} comandos.
\item<3->O resultado do {\color{DarkGreen}\bf processamento} de um 
{\color{DarkRed}\bf ficheiro} {\color{blue}\bf TeX} 
(extensão '{\color{blue}\bf .tex}')
é um {\color{DarkRed}\bf ficheiro} '{\color{blue}\bf .dvi}'.
(''{\color{DarkGreen} {\bf D}e{\bf V}ice {\bf I}ndependent}'') 
que contém as {\color{DarkRed}\bf localizações} de
{\color{red}\bf todos} os {\color{DarkGreen}\bf símbolos} e 
{\color{DarkGreen}\bf caracteres}. 
\uncover<4->{Em {\color{blue}\bf Unix}, a visualização dum {\color{DarkRed}\bf ficheiro} 
'{\color{blue}\bf .dvi}' faz-se com o programa '{\color{red}\bf xdvi}'.}
\item<5->A partir dos {\color{DarkRed}\bf ficheiros} '{\color{blue}\bf .dvi}' 
pode fazer-se a sua {\color{red}\bf conversão} 
para formatos específicos: '{\color{red}\bf .pdf}' ({\color{blue}\bf dvipdf}), 
'{\color{red}\bf .ps}' ({\color{blue}\bf dvips}), etc..
\item<6->Para {\color{DarkGreen}\bf definir} e {\color{DarkGreen}\bf gerir} 
as {\color{DarkRed}\bf fontes tipográficas} em {\color{blue}\bf TeX}, 
Donald Knuth introduziu a linguagem '{\color{red}\bf Metafont}'.
\item<7->Presentemente existem ao dispor dos utilizadores, para além das
usuais {\color{blue}\bf famílias de fontes}, fontes cobrindo quase todos
os tipos de escrita.
\end{itemize}
\end{frame}

\begin{frame}
\frametitle{{\color{blue}\bf Introdução}}
\begin{itemize}
\item<1->Um dos aspectos {\color{DarkGreen}\bf mais importantes} 
do {\color{blue}\bf TeX}
é o seu {\color{blue}\bf sistema de macros} com sua 
{\color{DarkRed}\bf enorme flexibilidade}.
\item<2->Graças a esta flexibilidade o {\color{blue}\bf TeX} tornou-se
a {\color{red}\bf base} para sistemas {\color{DarkRed}\bf mais fáceis} de manipular.
\item<3->Esses {\color{DarkGreen}\bf sistemas} são basicamente 
{\color{red}\bf pacotes} 
com um grande número de {\color{blue}\bf macros} predefinidas que 
permitem uma interacção {\color{DarkGreen}\bf mais agradável} 
com o utilizador.
\item<4->Muitos ambientes de {\color{DarkRed}\bf formatação complexa} 
encontram-se {\color{DarkRed}\bf definidos} nessas 
{\color{blue}\bf macros} permitindo gerir, com poucos comandos,
situações que, em {\color{blue}\bf TeX}, exigiriam um trabalho
muito mais árduo.
\item<5->No entanto, {\color{DarkRed}\bf não se pense} 
que trabalhar directamente em {\color{blue}\bf TeX} é 
'{\color{DarkGreen}\bf ter de definir tudo}'. 
Muito antes pelo contrário, quando se deseja 
uma certa {\color{DarkGreen}\bf liberdade de formatação},
a opção pela escrita em {\color{blue}\bf TeX} é, muito provavelmente,
a {\color{DarkRed}\bf escolha mais conveniente}.
\end{itemize}
\end{frame}

\begin{frame}
\frametitle{{\color{blue}\bf Introdução}}
\begin{itemize}
\item<1->O {\color{DarkGreen}\bf sistema de macros} {\color{red}\bf mais importante} 
construído sobre {\color{blue}\bf \TeX} é o {\color{blue}\bf \LaTeX}.
Foi desenvolvido no início dos {\color{blue}\bf anos 80s} pelo 
matemático {\color{blue}\bf Leslie Lamport} 
(SRI International -- iniciado pela Stanford University).
\item<2->Por sua vez um grande número de {\color{DarkGreen}\bf blocos adicionais} 
('{\color{DarkGreen}\bf packages}')
têm sido acrescentados colocando, ao dispor dos interessados, 
{\color{DarkRed}\bf funcionalidades} que permitem 
uma {\color{red}\bf elevada qualidade} do produto final.
\item<3->Actualmente, quase todas as {\color{DarkGreen}\bf sociedades científicas} 
e {\color{DarkGreen}\bf editoras} fornecem um {\color{DarkRed}\bf pacote específico} 
que tem definidas as {\color{blue}\bf macros} das 
{\color{blue}\bf normas editoriais} de formatação das revistas. % de que são responsáveis.
\par\uncover<4->{{\bf Exemplos:} {\color{blue}\bf amsmath} pacote de 
{\color{blue}\bf macros} da {\color{DarkRed}\bf American Mathematical Society} 
(\AmS) (pacote {\color{blue}\bf \AmS\TeX}) e o {\color{blue}\bf revtex} 
da {\color{DarkRed}\bf American Physical Society}}.
\item<5->Existem ainda pacotes com {\color{DarkGreen}\bf funcionalidades específicas} 
das {\color{DarkRed}\bf línguas}, dos {\color{DarkRed}\bf conjuntos de caracteres} 
a utilizar, etc..
\end{itemize}
\end{frame}

\begin{frame}
\frametitle{{\color{blue}\bf O Meu Primeiro Texto em \TeX\ e em \LaTeX}
{\color{black}\bf ('(La)TeX\_Prog\_01.tex')}}
\ \vskip -7mm\par 
\uncover<1->{No dois exemplos que se seguem mostra-se como se escreve
um ficheiro em {\color{blue}\bf TeX} e outro em {\color{blue}\bf LaTeX}.
Como se pode ver, quer o {\color{DarkRed}\bf texto} quer as 
{\color{DarkRed}\bf fórmulas}, são escritos exactamente da mesma maneira.}
\vskip -3mm
\begin{columns}
\column{.50\textwidth}
\begin{center}\uncover<2->{{\color{DarkRed}\bf \TeX}}\end{center}
\column{.50\textwidth}
\begin{center}\uncover<2->{{\color{DarkRed}\bf \LaTeX}}\end{center}
\end{columns}
\par\vskip 1mm
\begin{columns}
\column{.50\textwidth}
\hskip 4mm$\uncover<3->{\backslash${\color{blue}\bf magnification}=$\backslash${\color{DarkGreen}\bf magstep1}}
\par\hskip 4mm\uncover<9->{Bom dia a todos!}
\par\hskip 4mm\uncover<9->{Seja a express$\backslash$${\color{red}\bf \tilde{\ }}${\color{red}\bf \{}a{\color{red}\bf \}}o:}
\par\hskip 4mm\uncover<10->{$\backslash${\color{blue}\bf par}}
\par\hskip 4mm
\uncover<11->{\$$\backslash${\color{blue}\bf int}\_0$\hat{\ }$\{2$\backslash${\color{blue}\bf pi}\}$\backslash${\color{blue}\bf sin}(x) dx = 0\$}
\par\hskip 4mm\uncover<7->{$\backslash${\color{DarkRed}\bf end}}
\column{.50\textwidth}
\hskip 2mm$\uncover<3->{\backslash${\color{blue}\bf documentclass}[{\color{DarkGreen}\bf 12pt}]}\uncover<5->{\{{\bf article}\}}
\par\hskip 2mm\uncover<7->{$\backslash${\color{DarkRed}\bf begin}\{{\bf document}\}}
\par\hskip 2mm\uncover<9->{Bom dia a todos!}
\par\hskip 2mm\uncover<9->{Seja a express$\backslash$${\color{red}\bf \tilde{\ }}${\color{red}\bf \{}a{\color{red}\bf \}}o:}
\par\hskip 2mm\uncover<10->{$\backslash${\color{blue}\bf par}}
\par\hskip 2mm
\uncover<11->{\$$\backslash${\color{blue}\bf int}\_0$\hat{\ }$\{2$\backslash${\color{blue}\bf pi}\}$\backslash${\color{blue}\bf sin}(x) dx = 0\$}
\par\hskip 2mm\uncover<7->{$\backslash${\color{DarkRed}\bf end}\{{\bf document}\}}
\end{columns}
\begin{itemize}
\item<4->De modos diferentes, ambos se iniciam com o {\color{DarkGreen}\bf tipo de letra}.
\item<6->No caso do {\color{blue}\bf LaTeX} é indicada a 
{\color{DarkRed}\bf classe de texto} ({\bf article}).
\item<8->Em {\color{blue}\bf LaTeX} o texto está entre 
'$\backslash${\color{blue}\bf begin}' e 
'$\backslash${\color{blue}\bf end\{document\}}'; em 
{\color{blue}\bf TeX} termina apenas com o comando
'{\color{blue}\bf $\backslash$end}'.
\end{itemize}
\end{frame}

\begin{frame}
\frametitle{{\color{blue}\bf Exemplos de LaTeX}\par
{\color{black}\bf 'LaTeX\_Introd.tex', 'LaTeX\_Beamer.tex' e\par
'apssamp.tex' (exemplo de revtex)}}
\begin{itemize}
\item<1->Para finalizar esta pequena {\color{DarkGreen}\bf apresentação} de
{\color{blue}\bf\TeX}\ irei apresentar {\color{red}\bf três} exemplos da utilização de
{\color{blue}\bf\LaTeX}.
\item<2->O primeiro '{\color{DarkGreen}\bf LaTeX\_Introd.tex}'
é um pequeno texto que explica os {\color{DarkRed}\bf comandos básicos} 
de {\color{blue}\bf \LaTeX}\ e que os
{\color{DarkRed}\bf exemplifica} no seu {\color{blue}\bf próprio código}, 
os resultados que se obtêm.
\item<3->O segundo é o código dos '{\color{DarkRed}\bf slides}' desta aula.
Designei esse ficheiro por '{\color{blue}\bf LaTeX\_Beamer.tex}' uma vez
que a {\color{red}\bf classe} utilizada para os escrever foi '{\color{blue}\bf beamer}'.
\item<4->A pasta '{\color{DarkRed}\bf IncImages}' contém exemplos de integração de
  {\color{DarkGreen}\bf imagens} '{\color{blue}\bf .jpg}',
  '{\color{blue}\bf .png}' e '{\color{blue}\bf .pdf}'.
  Os respectivos ficheiros devem ser compilados com o comando '{\color{DarkGreen}\bf pdflatex}'.
\item<5->O terceiro é o código do exemplo que consta no pacote do 
'{\color{blue}\bf revtex4-1}' da {\color{DarkRed}\bf American Physical Society}.
\end{itemize}
\end{frame}

\end{document}
