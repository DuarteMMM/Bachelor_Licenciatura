\documentclass[a4paper,12pt]{article}

\usepackage[latin1,utf8]{inputenc}       % Tipos de caracteres
\usepackage[portuges]{babel}             % Português
\usepackage[a4paper,portrait]{geometry}  % Tipo de papel
\usepackage{amsmath}                     % Extensões da American Mathematical Society
\usepackage{multicol}                    % Para tratar colunas multiplas
\usepackage{makeidx}                     % Para fazer índices
\usepackage{color}                       % Para tratamento da cor
\usepackage{fancyhdr}                    % Para cabeçalhos
\usepackage{url}                         % Para tratar endereços 'url'
\usepackage{cite}

%%%% Verificar:
%\usepackage{ucs}
%\usepackage{utf8}
%\usepackage[latin1]{inputenc}
%\usepackage[applemac]{inputenc}         % Para o Mac

%%%%%%%%%%% Predefinições de LaTeX do Tamanho da Página %%%%%%%%%%

\oddsidemargin = 31pt                    % Margem do lado esquerdo: 31pt
\topmargin = 20pt                        % Margem superior: 20pt  
\headheight = 12pt                       % Tamanho do 'header': 12pt 
\headsep = 25pt                          % Espaço entre o 'header' e o texto: 25pt
\textheight = 592pt                      % Altura do texto: 592pt
\textwidth = 390pt                       % Largura do texto: 390pt
\marginparsep = 10pt                     % Espaço entre margem esquerda e o texto: 10pt
\marginparwidth = 35pt                   % Margem esquerda: 35pt
\footskip = 30pt                         % Espaço entre o texto e o 'footer': 30pt

\hyphenation{asso-ciada}
\input epsf

\makeindex

\begin{document}

\title{\bf Algumas Notas Básicas sobre \LaTeX}
\author{Samuel Eleutério \\ 
{\small sme@tecnico.ulisboa.pt} \\ 
Departamento de Física \\ 
Instituto Superior Técnico \\
Universidade de Lisboa}
\date{Dezembro de 2008 (Revisto em 2019)}

\maketitle

\begin{abstract}
Procura-se com esta pequena nota pôr ao dispor dos
alunos de Programação do Mestrado em Engenharia Física Tecnológica
alguns exemplos e informações úteis para a escrita de ficheiros
em \TeX / \LaTeX.
\par
Este texto foi elaborado no sentido de ser uma breve introdução 
ao \LaTeX\ e procura sê-lo pela análise conjunta do seu código e
da sua visualização.\,\,Pretende-se assim que ele seja o exemplo do 
que ele próprio descreve. Por isso, uma parte significativa dos comandos aqui 
referidos pode ser encontrada no código do texto, 
aconselha-se, pois, que a sua leitura seja acompanhada pela visualização 
desse mesmo código fonte.
\end{abstract}

\pagebreak

\tableofcontents

\pagebreak

\section*{Introdução}

É objectivo desta pequena nota exemplificar as situações mais usuais
que se colocam aos utilizadores de \TeX\ e \LaTeX\ na escrita de textos 
científicos.

Para além das obras originais de Donald Knuth\cite{Knuth:1984}
sobre \TeX\ e de Leslie Lamport\cite{Lamport:1986} sobre \LaTeX, encontra-se
disponível na internet uma bibliografia variada de 
excelente qualidade.

Para uma primeira experiência sobre \TeX\ / \LaTeX, bem como 
para posteriores consultas, o 
manual prático escrito por Michael Doob\cite{Doob}
''{\it A Gentle Introduction to \TeX}'', fornece uma boa
base de trabalho. Para além dos comandos básicos e de bastantes
exemplos, apresenta
uma lista razoavelmente completa dos símbolos matemáticos.

A edição da Wikibooks de \LaTeX\cite{LaTeX:Wiki} é 
um bom elemento de consulta disponível na internet. Dispõe ainda
de uma lista actualizada e minimamente documentada 
dos {\it packages} disponíveis.

O TUG (Indian \TeX\ Users Group), editou um pequeno manual
de \LaTeX\ {\it LaTeX Tutorials - A Primer}\cite{TUG:2003} e
um conjunto de slides cobrindo as principais funcionalidades
do sistema \LaTeX\ intitulado {\it Online Tutorials on LaTeX}\cite{TUG:2000}
em que é feita uma apresentação
sintética dos principais comandos.

Uma descrição detalhada das funcionalidades disponíveis no pacote da
American Mathematical Society (\AmS) pode ser encontrada em 
''{\it An Introduction to Mathematical Document Production 
Using \AmS\LaTeX}''\cite{Eveson} da autoria de Simon Eveson
(Universidade de York).

Para a escrita de textos científicos em Física é referência incontornável
o pacote '{\it revtex}'\cite{RevTeX}: conjunto de macros para \LaTeX2$\epsilon$
utilizado para publicação nos jornais da American Physical Society (APS) 
e do American Institute of Physics (AIP).

Finalmente, outros instrumentos muito cómodos na preparação de
documentação são as '{\it Reference Cards}' do ambiente em que
se está a trabalhar. Existem disponíveis na rede '{\it Reference Cards}'
para \TeX\cite{RefCard:TeX}, \LaTeX\cite{RefCard:LaTeX} e ainda 
para \AmS\TeX\cite{RefCard:AmSTeX} e 
para \AmS\LaTeX\cite{RefCard:AmSLaTeX}.

\break 

\section{Modo Texto}\index{Modo Texto}

Nesta secção vão ser apresentadas algumas das principais 
funcionalidades que se encontram ao dispor do utilizador 
para a escrita de textos.

\subsection{Classes de \LaTeX}\index{Classes de \LaTeX}
\index{Classes|see{Classes de \LaTeX}}\index{documentclass|see{Classes de \LaTeX}}

Quando se inicia um ficheiro em \LaTeX\ devemos indicar 
na primeira linha não comentada a sua '{\bf classe}'
e o '{\bf tamanho da letra}'\index{Tamanho da Letra} a utilizar. 
Ao indicar-se a '{\bf classe}',
está a optar-se por um determinado tipo de formato predefinido; ao indicar-se
o '{\bf tamanho da letra}' está a definir-se qual o tamanho básico que se
pretende utilizar. As alterações de tamanho de letra, feitas 
posteriormente ao longo do texto, devem ter um carácter relativo. 
Assim, se pretendermos
reduzir ou aumentar o tamanho global, as alterações far-se-ão de um
modo coerente. Exemplo: 
\vskip 2mm\par
$\backslash$documentclass[12pt]\{article\}
\vskip 2mm\par
Na referência\cite{LaTeX:Wiki} podem encontrar-se as classes predefinidas
em \LaTeX.
Note-se que qualquer utilizador poderá criar as suas próprias classes 
a partir das classes existentes.

\subsection{Organização do Texto}\index{Texto!Organização}
\index{Texto!Blocos}

A organização do texto é feita por blocos que se subdividem
em parcelas cada vez menores. A parcela maior é a '{\bf part}',
depois o '{\bf chapter}' até à mais pequena que é o
'{\bf subparagraph}'. Na marcação de cada um destes blocos
é feita a atribuição de um título.

Por exemplo, no caso deste texto não se quis que a seccão 'Introdução'
estivesse numerada como as outras, então foi marcada por uma estrela '*'
(ver código '.tex').

\subsection{Letras Acentuadas e Indicações Regionais}
\index{Acentos}\index{Internacionalização}

Um aspecto muito importante a ter em conta é a acentuação dos caracteres.
Quando o \TeX\ foi desenvolvido não existia nenhum mecanismo de incorporação
dos acentos devidamente estruturado. As soluções existentes na altura
eram bastante deficientes e dependiam das máquinas em que se
trabalhava. Por isso, a única opção razoável para se ter um sistema
que pudesse funcionar em qualquer computador era restringir os caracteres
utilizados aos 128 primeiros caracteres do ASCII.

Deste modo, os acentos e outras marcas a inserir deveriam ser feitos
por comandos próprios. Tal ainda hoje deverá ser feito em \TeX\ e 
em \LaTeX\ caso não se indiquem os tipos específicos que se estão
a usar. A título de exemplo apresenta-se a seguir uma tabela com
as marcas mais usuais (acentos e cedilhas):

\begin{table}[h]
\label{TabAcentos}
\caption{\bf Tabelas dos Acentos}
\begin{center}
\begin{tabular}{l|c|l}  \hline
\multicolumn{1}{c}{\bf Marcas} & \multicolumn{1}{|c|}{\bf Comando} & 
\multicolumn{1}{c}{\bf Exemplo} \\ \hline
Acento agudo         & $\backslash$'                  & caf\'e        \\ \hline
Acento agudo num 'i' & $\backslash$'\{$\backslash$i\} & f\'{\i}sica   \\ \hline
Acento grave         & $\backslash$`                  & \`a           \\ \hline
Acento circunflexo   & $\backslash\hat{\ }$           & c\^amara      \\ \hline
Trema                & $\backslash$"                  & Schr\"odinger \\ \hline
Til                  & $\backslash\tilde{\ }$         & c\~ao         \\ \hline
Cedilha              & $\backslash$c                  & ca\c{c}a      \\ \hline
\end{tabular}
\end{center}
\end{table}

Em \LaTeX, a declaração dos '{\it packages}' 
''{\bf inputenc}'' e ''{\bf babel}'', 
no início deste ficheiro, permite-nos a utilização do formato '{\bf utf-8}' 
com as especificações do português. Outras indicações sobre a utilização 
do sistema de caracteres 'Unicode' podem ser encontradas na literatura.
\index{Internacionalização}\index{Internacionalização!Unicode}
\index{Internacionalização!UTF-8}\index{Internacionalização!Português}
\index{Packages}\index{Packages!inputenc}\index{Packages!babel}

\subsection{Formatação Básica}

Não é objectivo do sistema \TeX\ reproduzir, em tempo real, a escrita
de um texto. Por isso, não há necessidade de uma formatação cuidada do texto
nos ficheiros '{\it .tex}'. No entanto, há algumas regras a ter em conta:

\begin{itemize}
\item {\bf Comentários:}\index{Comentário} Iniciam-se por '\%'.
\item {\bf Parágrafo:}\index{Texto!Parágrafo} é um bloco de texto
que começa e acaba numa, ou mais, linhas em branco. Em alternativa,
pode iniciar-se um parágrafo com o comando '$\backslash${\bf par}'.
Um erro muito frequente é a introdução de linhas em branco
no ficheiro, para melhor visualização do código fonte, ignorando o
seu papel na formatação do mesmo.
\item {\bf Indentação:}\index{Texto!Indentação} 
Por defeito, os parágrafos são 
indentados (excepto o primeiro). No entanto, ela pode ser 
inibida com '$\backslash$noindent'.
\end{itemize}

\subsection{Listas}
\index{Listas}\index{Listas!itemize}\index{Listas!enumerate}\index{Listas!item}

A enumeração de tópicos, como a que se encontra na sub-secção 
'{\bf Formatação Básica}', designa-se por '{\bf itemize}'.
A sua delimitação é feita pelos comandos '$\backslash${\bf begin}\{{\it itemize}\}' e
$\backslash${\bf end}\{{\it itemize}\}'. A inserção de cada elemento na
lista é precedida com comando '$\backslash${\bf item}'.

É igualmente possível escolher listagens em que a enumeração
tem associada um contador. Tais listas são declaradas como
'{\bf enumerate}' (ver exemplo na sub-secção '{\bf Tabelas}').

\subsection{Tabelas}
\index{Tabelas}\index{table|see{Tabelas}}\index{tabular}

Na construção de uma tabela são consideradas duas partes:
\begin{enumerate}
\item O '{\bf tabular}': que corresponde à tabela propriamente
dita e que inclui a quadrícula e os elementos nela inseridos;
\item A '{\bf table}': que corresponde à moldura em que está
contido o '{\bf tabular}'. É ainda constituída
por uma legenda, pela indicação da sua localização no texto 
e por um '{\bf label}' que permiti referi-la.
\end{enumerate}

Como exemplo, ver o código fonte da tabela
criada na sub-secção ''{\bf Letras Acentuadas e Indicações Regionais}''.
Essa tabela contém um '{\bf tabular}' definido pelo comando:
\vskip 1.5mm\par
'$\backslash${\bf begin}\{{\it tabular}\}\{l$|$c$|$l\} $\backslash${\bf hline}'
\vskip 1.5mm\par\noindent
ao qual se segue uma sequência de caracteres entre chavetas que indica o
número de colunas e as caracteriza: '{\bf l}'
significa que a primeira coluna será alignada ao lado esquerdo,
a '$|$' indica que vai existir uma linha vertical entre a primeira e a 
segunda coluna. '{\bf c}' indica que os elementos da segunda 
coluna deverão ser centrados, etc. Finalmente, note-se que
antes da primeira coluna e depois da última não existem '$|$',
isso faz com que o '{\bf tabular}' fique aberto à esquerda e
à direita.

O comando '$\backslash${\bf hline}' indica que irá existir uma linha
horizontal antes da primeira linha do '{\bf tabular}'.

\subsection{Figuras}\index{Figuras}

Existe uma razoável diversidade de maneiras de inserir
figuras no texto. Iremos aqui mostrar dois exemplos mas,
muitos outros podem ser encontrados na literatura.
Na 'figura 1' é apresentado o gráfico de uma função.

\begin{figure}[h]
\caption{Simulação da exponencial pelo método de Euler}
\begin{center}
\epsfysize=5.0truecm\epsffile{imagens/Figura01.eps}
\end{center}
\end{figure}

\par
%\hbox{\hskip 1.0truecm\vbox to 1.2 cm{\hsize=.60\hsize \hskip 1.1cm
\hbox{\vbox to 1.2 cm{\hsize=.55\hsize
Como segundo exemplo é apresentada uma imagem inserida à direita de
um bloco de texto.}
\vbox to 1.3 cm{\vskip 0.1truecm
\noindent\hskip 0.5truecm\epsfysize=1.2truecm\epsffile{imagens/wwwist.eps}}}
\vskip 1.5mm\par
No primeiro caso foi definida uma '{\bf figure}' pelo que tem uma
legenda e um posicionamento; no segundo caso apenas se inseriu
a figura no interior de uma '{\bf box}'.

A descrição do sistema de '{\bf boxes}' ultrapassa a dimensão
deste texto. Para a sua compreensão poderá consultar-se a bibliografia.\index{Boxes}

\subsection{Espaçamento}\index{Espaçamento}

O control do espaçamento vertical pode ser feito através de comandos 
como o '$\backslash${\bf bigskip}' ou o '$\backslash${\bf smallskip}'; no
que diz respeito ao espaçamento horizontal são apresentados
alguns exemplos no quadro que se segue:\index{Espaçamento!Vertical}

\begin{center}\index{Espaçamento!Horizontal}
\begin{tabular}{l|c|l} 
\multicolumn{3}{c}{\bf Tabelas de Espaçamentos}            \\ \hline
{\bf Comando}           & {\bf Abreviatura} & {\bf Efeito} \\ \hline
(sem espaçamento)       &                   & XX           \\ \hline
$\backslash$thinspace   & $\backslash$,     & X$\,$X       \\ \hline
$\backslash$medspace    & $\backslash$:     & X$\:$X       \\ \hline
$\backslash$thickspace  & $\backslash$;     & X$\;$X       \\ \hline
$\backslash$quad        &                   & X$\quad$X    \\ \hline
$\backslash$qquad       &                   & X$\qquad$X   \\ \hline
\end{tabular}
\end{center}

É igualmente possível dar espaçamentos horizontais e verticais
com dimensões fixas (cm, mm, pt, in, etc.) para tal usam-se
os comandos '$\backslash${\bf hskip}' e '$\backslash${\bf vskip}' 
seguidos dos respectivos valores numéricos 
(exemplos: '$\backslash${\bf hskip} 1.5mm',
'$\backslash${\bf vskip} 3pt').

\subsection{Caracteres de Comando}
\index{Caracteres!Especiais}
\index{Caracteres!de Comando|see{Especiais}}

Um certo número de caracteres têm, em \TeX, um significado diferente
do seu valor normal, isto é, servem para introduzir ou declarar 
instruções. Por isso, a sua introdução, em texto, tem de ser 
indicada de um modo especial.

\begin{table}[h]
\label{TabSpecial}
\caption{\bf Tabelas de Caracteres Especiais}
\begin{center}
\begin{tabular}{c|c|l}  \hline
\multicolumn{1}{c|}{\bf Caracter} & {\bf Escrita} & 
\multicolumn{1}{c}{\bf Significado} \\ \hline
$\backslash$            &  \$$\backslash$backslash\$ & Início de instruções        \\ \hline
\{                      &  $\backslash$\{            & Abrir agrupamento           \\ \hline
\}                      &  $\backslash$\}            & Fechar agrupamento          \\ \hline
\%                      &  $\backslash$\%            & Comentário                  \\ \hline
\&                      &  $\backslash$\&            & Comando de alinhamento      \\ \hline
\~{}                    &  $\backslash$\~{}\{\}      & Espaço não separável        \\ \hline
\$                      &  $\backslash$\$            & Modo matemático             \\ \hline
\^{}                    &  $\backslash$\^{}\{\}      & Expoente em modo matemático \\ \hline
\_{}                    &  $\backslash$\_{}\{\}      & Índice em modo matemático   \\ \hline
\#                      &  $\backslash$\#            & Substituição de símbolos    \\ \hline
\end{tabular}
\end{center}
\end{table}

\subsection{Notas de Fim de Pagina e Notas à Margem}
\index{Notas}\index{Notas!footnote}\index{Notas!marginpar}

As notas de fim de página\footnote{As notas de fim de
página são também designadas na literatura por {\it footnotes}.}
são inseridas directamente no 
lugar em que a chamada é feita. O comando usado é 
'$\backslash${\bf footnote}\{...\}', em que o bloco contém o texto
completo da nota.
%\index{Notas!Fim de Página}\index{footnote|see{Notas}}

Um outro tipo de nota especialmente útil para destacar 
\marginpar{{\bf Atenção:} {\small Nota à direita}}
informações ou para referir alterações no texto
quando se trabalha em colaboração com outras pessoas 
é a nota à margem cujo comando é
'$\backslash${\bf marginpar}\{...\}'. Como no caso anterior,
dentro das chavetas deverá ser colocado o 
texto. 
%\index{Notas!À margem}

É igualmente possível inserir notas em zonas específicas do
texto, por exemplo, junto de tabelas ou de figuras. Para tal,
deve recorrer-se ao conceito de 'minipágina' 
({\it minipage}).
\index{minipage}\index{Mini-página|see{minipage}}

\subsection{Tamanho dos Caracteres}

Como se disse o tamanho dos caracteres deve ser definido em 
relação ao tamanho base definido no início, na tabela
seguinte podem ver-se alguns dos tamanhos definidos:

\begin{table}[h]
\label{TabSize}
\caption{\bf Tamanho dos Caracteres}
\begin{center}
\begin{tabular}{l|l} \hline
\multicolumn{1}{c|}{\bf Comando} & \multicolumn{1}{c}{\bf Exemplificação}  \\ \hline
$\backslash$scriptsize  & {\scriptsize Isto é o tamanho 'scriptsize'}   \\ \hline
$\backslash$footnotesize  & {\footnotesize Isto é o tamanho 'footnotesize'}   \\ \hline
$\backslash$small  & {\small Isto é o tamanho 'small'}   \\ \hline
$\backslash$normalsize  & {\normalsize Isto é o tamanho 'normalsize'}   \\ \hline
$\backslash$large  & {\large Isto é o tamanho 'large'}   \\ \hline
\end{tabular}
\end{center}
\end{table}

\subsection{Índice (Table of Contents)}\index{Índice}\index{Table of Contents}

Para criar um índice (table of contents - TOC) basta
inserir o comando '$\backslash$tableofcontents' no sítio desejado.
Do índice farão parte os conteúdos das macros '$\backslash$chapter',
'$\backslash$section', '$\backslash$subsection', etc.
\par
Há, no entanto, que ter em conta que para o índice aparecer há
necessidade de executar duas vezes a compilação em \LaTeX:
na primeira vez o ficheiro com índice é criado e na segunda
vez ele é então integrado no texto.

\subsection{Índice Remissivo}\index{Índice Remissivo}

A instrução para a criação dum índice remissivo, 
'{\bf $\backslash$makeindex}', é feita no início do ficheiro, antes do
comando '$\backslash${\bf begin}\{{\it document}\}'.
\index{makeindex}

As instruções de criação deste índice são feitas pelo comando
'$\backslash${\bf index}' a que se segue a informação que se
pretende incluir. Nos casos mais simples da sua aplicação
é apenas indica a entrada do índice, 
'$\backslash${\bf index}\{{\it Índice}\}', ou de uma sub-entrada, 
'$\backslash${\bf index}\{{\it Índice!Sub\_Índice}\}'.
\index{Index}

Na execução, na {\it shell}, do comando {\bf latex}
é criado um ficheiro com a extensão '{\bf .idx}' que contém a informação
para a criação do índice.

Ainda na {\it shell} deve ser executado o
programa '{\bf makeindex} $<$nome.idx$>$'. Como resultado, 
são criados dois ficheiros um '{\bf .ilg}' e outro '{\bf .ind}' 
que contêm o índice do texto. 
Deve então executar-se novamente o comando {\bf latex} para ter
a sua correcta integração no texto.\index{Comandos!makeindex}

Ao longo do código deste texto podem ser encontradas várias
indicações de inclusão no índice.

\break

\section{Modo matemático}\index{Modo Matemático}

Na referência \cite{TUG:2000}, bem como noutra documentação, 
existem listas mais ou menos completas dos símbolos predefinidos
e das letras gregas.

O modo matemático é iniciado e terminado por um '\$'.
Os expoentes são introduzidos por um acento circunflexo: '$\hat{\ }$'
e os índices pela barra '\_'. No caso de serem constituídos por 
mais do que um caracter devem ser delimitados por chavetas \{...\}.
Exemplo:

\begin{multicols}{2}
\center{\bf Resultado}
\vskip 2mm
\par
$f(x_1) = x_1^2 - 5 x_1 + 6$
\par
\center{\bf \LaTeX (ou \TeX)}
\vskip 2mm
\$f(x\_1) = x\_1$\hat{\ }$2 - 5 x\_1 + 6\$
\end{multicols}

Note-se que quando se fala em modo matemático, isso é uma designação
muito genérica, que inclui toda uma simbologia que
normalmente é usada nas áreas das ciências e das engenharias.

\subsection{Exemplifição de Expressões Matemáticas}

\subsubsection{Representações de Matrizes}
\index{Modo Matemático!Matrizes}

\begin{center}
$\begin{matrix}
\alpha_{11} & \alpha_{12}\\
\alpha_{21} & \alpha_{22}
\end{matrix}$
\hskip 20mm
$\begin{bmatrix}
\alpha_{11} & \alpha_{12}\\
\alpha_{21} & \alpha_{22}
\end{bmatrix}$
\hskip 20mm
$\begin{vmatrix}
\alpha_{11} & \alpha_{12}\\
\alpha_{21} & \alpha_{22}
\end{vmatrix}$
\end{center}

\subsubsection{Combinações\cite{TUG:2003}}
\index{Modo Matemático!Combinações}

\[\left(\begin{array}{c} n \\ \\ r \end{array} \right) = 
\frac{n!}{r!(n-r)!}\]

\subsubsection{Equações e Fracções}
\index{Modo Matemático!Equações}\index{Modo Matemático!Fracções}

\begin{equation}
\begin{split}
f(x,y) & = {1 \over 6} \times (2\,x^2 + 3\,x\,y + x\,y) \\
       & = {1 \over 3}\,(x^2 + 2\,x\,y) \\
       & = {x^2 + 2\,x\,y \over 3}
\end{split}
\end{equation}

\subsubsection{Sistemas de Equações}
\index{Modo Matemático!Sistemas de Equações}

\hskip 10mm$\left\{
\begin{matrix}
x_o & = & A\,sen\,(\varphi)           \hfill  \\
v_o & = & A\,\omega_o\,cos\,(\varphi) \hfill 
\end{matrix}
\right .$

\subsubsection{Fracções\cite{TUG:2003}}
\index{Modo Matemático!Fracções}

\begin{equation}
\nonumber
  x = a_0 + \frac{1}{\displaystyle a_1
          + \frac{1}{\displaystyle a_2
          + \frac{1}{\displaystyle a_3 + a_4}}}
\end{equation}

\subsubsection{Somatórios}
\index{Modo Matemático!Somatórios}

\hskip 10mm
$f(x) = \sum_{n = - \infty}^{\infty} c_n\,e^{in\pi x/L}$
\hskip 20mm
$\displaystyle
f(x) = \sum_{n = - \infty}^{\infty} c_n\,e^{in\pi x/L}$

$$f(x) = \sum_{n = - \infty}^{\infty} c_n\,e^{in\pi x/L}$$

\subsubsection{Integrais}
\index{Modo Matemático!Integrais}

\hskip 10mm
$\int_{-\infty}^{\infty} e^{ax^2} dx = \sqrt{\pi \over a}$
\hskip 30mm
$\displaystyle
\int_{-\infty}^{\infty} e^{ax^2} dx = \sqrt{\pi \over a}$

$$\int_{-\infty}^{\infty} e^{ax^2} dx = \sqrt{\pi \over a}$$

\break
\section{Como Usar e Instalar}

\subsection{Como utilizar o \TeX\ e o \LaTeX}

Os ficheiros com o código \TeX\ (ou \LaTeX) podem ser
criados com qualquer editor que não introduza elementos de
formatação próprios ('emacs', 'gedit', 'NotePad', etc.) e devem
ter a extensão '{\bf .tex}'. O processamento desses ficheiros é feito 
pelos programas '{\bf tex}' e '{\bf latex}' seguidos do nome do 
ficheiro.\index{Editores}\index{Comandos!tex}\index{Comandos!latex}

O resultado do processamento por estes programas é um ficheiro
'{\bf .dvi}' ({\bf D}e{\bf V}ice {\bf I}ndependent) 
que contém as localizações de todos os símbolos e caracteres
a imprimir. Este ficheiro pede ser visualizado por intermédio do programa
'{\bf xdvi}'. É de notar que não é necessário executá-lo uma segunda
vez para visualizar as actualizações: logo que o novo ficheiro 
'{\bf .dvi}' é criado ele faz a sua actualização automática no écran.
\index{Comandos!xdvi}

Para transformar este ficheiro '{\bf .dvi}' noutros tipos há que
usar programas que transformam este formato noutros. Os mais 
frequentes são:

\begin{itemize}
\item '{\bf dvipdf}': \index{Comandos!dvipdf}
que o transforma para formato '{\bf .pdf}'.
Note-se que é possível fazer a passagem directamente de \LaTeX\ 
para '{\bf .pdf}' com o programa '{\bf pdflatex}'. Os ficheiros
'{\bf .pdf}' podem ser visualizados com um dos programas: 
'acrobat reader', 'xpdf', 'evince', etc.
\item '{\bf dvips}': \index{Comandos!dvips}
que o transforma para o formato '{\bf .ps}'
(postscript), utilizado por muitas impressoras. A sua visualização 
pode ser feitas com o programa '{\bf gv}' ('{\bf ghostview}')
disponível para 'unix' e para 'Windows'.
\end{itemize}

\subsection{Como instalar o \TeX\ e o \LaTeX}
\index{Instalação}

\index{Instalação!Linux}
As diferentes instalações de unix (linux) têm, nos seus pacotes,
versões completas de \TeX. No caso de não estarem instaladas, tal
pode ser feito com os respectivos gestores de pacotes.

\index{Instalação!Microsoft Windows}
Para a instalação em ambiente Microsoft Windows deverá ser feita 
a instalação de uma implementação de \TeX\ disponível.
Para tal poderá ser usado o '{\bf MiKTeX}' (http://miktex.org/).
A sua instalação é bastante simples, consiste em fazer o download
dos ficheiros para uma directoria e em seguida correr o programa
de '{\bf setup}'. Os programas são depois usados numa janela de 'DOS'
dum modo idêntico ao descrito no ponto anterior.

Note-se que existem ambientes mais ou menos agradáveis em que
se pode trabalhar em \LaTeX\ (LyX, kile, etc.). Deixa-se ao critério de cada um fazer
as suas opções pessoais, no entanto, é altamente recomendado que
independentemente da escolha que se faça, o utilizador tenha
alguma experiência de trabalho em modo texto.


%%%%%%%%%%%%%%%%%%%%%%%%%%%%%%%%%%%%%%%%%%%%%%%%%%%%%%%%%%%%%%%%%%%%%

\printindex

\begin{thebibliography}{99}

\bibitem {Knuth:1984}
Donald E. Knuth. {\it The TeXbook}.
Addison-Wesley, Reading, Massachusetts: 1984. ISBN 0-201-13448-9.

\bibitem {Lamport:1986}
Leslie Lamport. {\it LaTeX: A Document Preparation System}. 
Addison-Wesley, Reading, Massachusetts: 2nd. ed., 1994. ISBN 0-201-52983-1.

\bibitem {Doob}
Michael Doob. {\it A Gentle Introduction to \TeX\ - A Manual for Self-study}.
Department of Mathematics. The University of Manitoba. Winnipeg. Manitoba.
Canada R3T 2N2.
\par [http://onlinebooks.library.upenn.edu/webbin/book/lookupname?key=
Doob\%2C Michael]

\bibitem {LaTeX:Wiki}
{\it LaTeX} by Wikibooks contributors.
\par [http://en.wikibooks.org/wiki/LaTeX]

\bibitem {TUG:2003}
TUG - Indian TeX Users Group.
{\it LaTeX Tutorials - A Primer}. Editor: E. Krishnan.
\par [http://www-h.eng.cam.ac.uk/help/tpl/textprocessing/ltxprimer-1.0.pdf]

\bibitem {TUG:2000}
TUG - Indian TeX Users Group.
{\it Online Tutorials on LaTeX}.
\par [http://amath.colorado.edu/documentation/LaTeX/tutorial/]

\bibitem {Eveson}
Simon Eveson. 
{\it An Introduction to Mathematical Document Production Using \AmS\LaTeX}.
Edited by Tony Sudbery.
\par [http://www-users.york.ac.uk/~spe1/texnotes07.pdf]

\bibitem {RevTeX}
\par [http://authors.aps.org/revtex4/]

\bibitem {RefCard:TeX}
Reference Card de TeX.\par
http://refcards.com/docs/silvermanj/tex/tex-refcard-a4.pdf

\bibitem {RefCard:LaTeX}
Reference Card de LaTeX.\par
[http://www.stdout.org/~winston/latex/latexsheet-a4.pdf]

\bibitem {RefCard:AmSTeX}
Reference Card de AmSTeX.\par
http://www.digilife.be/quickreferences/QRC/AMSTeX%20Reference%20Card%201.3.pdf

\bibitem {RefCard:AmSLaTeX}
Reference Card de AmSLaTeX.\par
http://refcards.com/docs/silvermanj/amslatex/LaTeXRefCard.v2.0.pdf

\end{thebibliography}

\end{document}

-----------> footnotes
